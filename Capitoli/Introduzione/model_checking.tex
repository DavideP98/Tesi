\section{Introduzione al model checking}
Il model checking è una tecnica finalizzata a dimostrare che un determinato
sistema, sia esso software o hardware, soddisfi determinate proprietà di
comportamento.
La pratica comune consiste nel modellare tale sistema con un linguaggio 
formale, ad esempio le reti di Petri, e di esprimere, solitamente in
formule LTL (\emph{Linear Time Temporal Logic}), una serie di 
proprietà di comportamento che il sistema deve soddisfare per
potersi definire corretto.

Un software di model checking prende quindi come input una rappresentazione
formale del sistema, o una sua conversione in un sistema di transizione,
ed una formula logica che esprime una certa proprietà.
Il software ritorna, nel caso in cui la formula non sia soddisfatta dal 
sistema, una \emph{traccia}, ossia un esempio di una possibile 
evoluzione del sistema in cui la formula non assume valore di verità.

\subsection{Introduzione a LTL}

\subsection{Il model checker Selt}
Selt è un software di model checking facente parte del pacchetto TINA,
(\emph{TIme petri Nets Analyzer}), un insieme di programmi per la
rappresentazione e l'analisi delle reti di Petri temporizzate.

Il paradigma di lavoro per l'utilizzo di Selt è il seguente: 
Si utilizza l'editor grafico Nd (\emph{NetDraw}) per disegnare una rete di
Petri, la si converte in un sistema di transizione usando \emph{Sift},
il quale restituisce un file binario in formato \texttt{.ktz}.
Selt prende in input il file generato e una formula LTL, 
nella quale compaiono come variabili proposizionali i nomi 
dei posti presenti nella rete. Selt provvede a verificare che
la formula passata in input sia soddisfatta per ogni possibile evoluzione
del sistema.

La figura \ref{fig:mutex} mostra una semplice rete di Petri che
modella una scenario in cui due processi si contendono l'accesso 
alla risorsa $P_1$.
I due processi in questione sono $\{P_0, t_0, P_3, t_1\}$ e 
$\{P_2, t_2, P_4, t_3\}$, in cui i posti $P_3$ e $P_4$ rappresentano le 
sezioni critiche.
Siamo interessati a verificare che l'accesso alla risorsa $P_1$ avvenga
in modo mutuamente esclusivo, 
per fare ciò codifichiamo la proprietà di mutua esclusione in una formula
LTL: $G \lnot (P_3 \land P_4)$ e la forniamo in input a Selt insieme al
sistema di transizione relativo alla rete.
Selt fornisce in output \texttt{TRUE} e il tempo impiegato per la verifica
della formula, che in questo caso risulta essere \texttt{0.001s}.

\begin{figure}[H]
    \centering
    \begin{tikzpicture}
        \node[place,
            minimum size=0.7cm,
            tokens=1,
            label={above:$P_1$}
            ] (p1) at (0,0) {};
        \node[place,
            minimum size=0.7cm,
            tokens=1,
            label={above:$P_0$}
            ] (p0) at (-2.5,0) {};
        \node[place,
            minimum size=0.7cm,
            tokens=1,
            label={above:$P_2$}
            ] (p2) at (2.5,0) {};
        \node[transition,
            minimum size=0.7cm,
            label={right:$t_0$}
            ] (t0) at (-2.5,-2) {};
        \node[transition,
            minimum size=0.7cm,
            label={right:$t_2$}
            ] (t2) at (2.5,-2) {};
        \node[place,
            minimum size=0.7cm,
            label={45:$P_3$}
            ] (p3) at (-2.5,-4) {};
        \node[place,
            minimum size=0.7cm,
            label={45:$P_4$}
            ] (p4) at (2.5,-4) {};
        \node[transition,
            minimum size=0.7cm,
            label={right:$t_1$}
            ] (t1) at (-2.5,-6) {};
        \node[transition,
            minimum size=0.7cm,
            label={right:$t_3$}
            ] (t3) at (2.5,-6) {};
        \draw[thick] 
        (p0) edge[post] (t0)
        (t0) edge[post] (p3)
        (p3) edge[post] (t1)
        (p2) edge[post] (t2)
        (t2) edge[post] (p4)
        (p4) edge[post] (t3)
        (p1) edge[post] (t0)
        (t1) edge[post] (p1)
        (t3) edge[post] (p1)
        (p1) edge[post] (t2)
        (t1) edge[post, bend left=40] (p0)
        (t3) edge[post, bend right=40] (p2);
    \end{tikzpicture}
    \caption{Accesso mutuamente esclusivo ad una risorsa condivisa}
    \label{fig:mutex}
\end{figure}
