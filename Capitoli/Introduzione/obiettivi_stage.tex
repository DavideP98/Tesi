\section{Obiettivi del progetto di stage}
L'esperienza di stage è stata incentrata sull'approfondimento di tematiche 
riguardandi il \emph{Model Checking}, un argomento già incontrato
nell'insegnamento di Metodi
Formali. In particolare, l'obiettivo principale del tirocinio è consistito 
nell'apprendimento del software di verifica automatica \emph{Spin}
e del suo linguaggio di specifica dei modelli: \emph{Promela}.

Si è anche fatto uso di vari strumenti presenti nel pacchetto software
\emph{TINA}, fra cui il model checker \emph{Selt}, usato come termine di
paragone per effettuare un confronto con Spin, sia in termini di differenze 
nell'utilizzo che in termini di prestazioni.
I software del pacchetto TINA, a differenza di Spin, sono incentrati
sull'analisi delle reti di Petri, una struttura trattata esaustivamente
nell'insegnamento di Metodi Formali, che ha anche svolto un ruolo centrale
nel tirocinio.
Ci si è infatti posti il problema di simulare in Promela, secondo diverse
metodologie, il comportamento delle reti di Petri e di effettuare verifiche di
proprietà su di esse.

% Inserire spiegazione sul  

Il confronto tra le prestazioni è avvenuto sperimentalmente: si sono
effettuate delle misurazioni sui tempi di esecuzione dei due programmi a
cui è stata richiesta la verifica della medesima proprietà
, espressa con una formula LTL, su dei modelli equivalenti.
Si è anche sviluppato un programma per effettuare l'espansione di una rete di
Petri, basandosi su una rete di partenza e su direttive fornite dall'utente.
Lo scopo del programma è quello di generare reti di Petri di grandi
dimensioni, allungando quindi i tempi di esecuzione di Spin e Selt,
cosicchè il confronto delle prestazioni risulti più significativo.






