\section{Introduzione alle reti di Petri}

Le reti di Petri sono una \emph{struttura matematica} ampiamente utilizzata 
per la modellazione dei sistemi distribuiti. Costituiscono un linguaggio
formale per rappresentare sistemi di processi concorrenti in cui
possono verificarsi conflitti, sincronizzazioni e scambi di messaggi.
Essendo una struttura definita in modo rigoroso, l'analisi del comportamento
dei sistemi modellati con reti di Petri può essere condotta avvalendosi di
strumenti matematici, al fine di verificare che il sistema goda di certe
proprietà.

\begin{definition}
\normalfont
Una rete di Petri è definita tramite una tripla $ N = (P, T, F) $ dove 
$P$ è l'insieme dei \emph{posti}, $T$ l'insieme delle \emph{transizioni}
e $F$ la relazione di \emph{flusso}, definita in questo modo: 
$F \subseteq (P \times T) \cup (T \times P) $.
\end{definition}

Dato che $P$ e $T$ sono insiemi disgiunti e che $F$ definisce solo archi dai
posti alle transizioni e viceversa, si può concludere che una rete di Petri
costituisce un grafo bipartito $G = (P \cup T , F) $.

\begin{definition}
\normalfont
Ciascun elemento $x \in P \, \cup \, T$ ha \emph{pre-elementi} e \emph{post-elementi}, 
denotati rispettivamente da $\leftidx{^\bullet}{x}{}$ e $x^{\bullet}$, così
definiti: 
$$ \leftidx{^\bullet}{x}{} = \{ y \in P \cup T : (y, x) \in F \},\quad
x^{\bullet} = \{ y \in P \cup T : (x, y) \in F \}$$

Ad esempio, nella rete mostrata nella figura \ref{fig:petri_1} vale che 
$\leftidx{^\bullet}{t_1}{} = \{P_2\}$, e $t_1^{\bullet} = \{P_1, P_3\}$.
\end{definition}

% Illustrazione rete di Petri
\begin{figure}[H]
    \centering
    \begin{tikzpicture}
        \node[place,
            minimum size=0.7cm,
            label={below:$P_1$}
            ] (p1) at (0,0) {};
        \node[transition,
            minimum size=0.6cm,
            label={45:$t_1$}
            ] (t1) at (-3,0) {};
        \node[place,
            minimum size=0.6cm,
            label={above:$P_2$}
            ] (p2) at (-4.5,1.5) {};
        \node[place,
            minimum size=0.6cm,
            label={below:$P_3$}
            ] (p3) at (-4.5,-1.5) {};
        \node[transition,
            minimum size=0.6cm,
            label={135:$t_2$}
            ] (t2) at (-6,0) {};
        \node[transition,
            minimum size=0.6cm,
            label={135:$t_3$}
            ] (t3) at (3,0) {};
        \node[place,
            minimum size=0.6cm,
            label={above:$P_4$}
            ] (p4) at (4.5,1.5) {};
        \node[place,
            minimum size=0.6cm,
            label={below:$P_5$}
            ] (p5) at (4.5,-1.5) {};
        \node[transition,
            minimum size=0.6cm,
            label={45:$t_4$}
            ] (t4) at (6,0) {};
        \draw[thick] 
        (t1) edge[post] (p1)
        (p2) edge[post, bend left=40] (t1)
        (t1) edge[post, bend left=40] (p3)
        (p3) edge[post, bend left=40] (t2)
        (t2) edge[post, bend left=40] (p2)
        (p1) edge[post] (t3)
        (p4) edge[post, bend right=40] (t3)
        (t4) edge[post, bend right=40] (p4)
        (p5) edge[post, bend right=40] (t4)
        (t3) edge[post, bend right=40] (p5);
    \end{tikzpicture}
    \caption{Esempio di rappresentazione grafica di una rete di Petri: i posti sono denotati da cerchi, le transizioni da quadrati}
    \label{fig:petri_1}
\end{figure}

E' possibile associare ad una rete una marcatura, ossia una funzione
\mbox{$m : P \to \mathbb{N}$}, ottenendo così un \emph{sistema}
$\Sigma = (N; m_0)$, dove $m_0$ viene
definita \emph{marcatura iniziale}.
Una marcatura costituisce lo stato globale del sistema, che evolve per mezzo dell'esecuzione, o \emph{scatto}, delle proprie transizioni.
Una transizione può scattare solo se essa è abilitata nella marcatura in
cui il sistema si trova.

\begin{definition}
\normalfont
Una transizione $t$ è \emph{abilitata} nella marcatura $m$, (denotato $m[t>$)
$\iff \forall p \in \leftidx{^\bullet}{t}{}, m(p) > 0$.

Lo scatto di una transizione comporta la generazione di una nuova marcatura $m'$,
denotato $m[t {\,>\,} m'$, dove $m'$ è esprimibile in funzione di $m$:

$$
m'(p) = \begin{cases}m(p) - 1 & se \; p \in \leftidx{^\bullet}{t}{}
\land p \notin t^{\bullet}\\
m(p) + 1 & se\; p \in t^{\bullet} 
\land p \notin \leftidx{^\bullet}{t}\\m(p) & altrimenti\end{cases}
$$

\end{definition}

\begin{definition}
\normalfont
Dato $U \subseteq T $ e una marcatura $m$, si definisce $U$ un insieme di transizioni \emph{indipendenti} $\iff$
$$
\forall t_1, t_2 \in U: t_1 \neq t_2 \implies
(\leftidx{^\bullet}{t_1}{} \cup t_1^{\bullet}) \cap 
(\leftidx{^\bullet}{t_2}{} \cup t_2^{\bullet})
$$

Si può inoltre definire $U$ come un insieme di transizioni 
\emph{concorrenti} $\iff$
$$ U \textrm{ è un insieme di transizioni indipendenti} \land \forall t
    \in U: m[t> 
$$

Nella figura \ref{fig:sistema_el} si può osservare un esempio di sistema e
di una sua possibile evoluzione. La marcatura iniziale 
$
m_0 =
\begin{bmatrix}
1 \, 1 \, 0 \, 0 \, 0
\end{bmatrix}^T
$
è rappresentata graficamente da dei pallini inseriti all'interno dei posti.
Si può notare che $a$ e $b$ sono transizioni concorrenti, è infatti
possibile che esse scattino in un unico passo di computazione:
$ 
\begin{bmatrix}
1 \, 1 \, 0 \, 0 \, 0
\end{bmatrix}^T
[\,\{a,b\}>
\begin{bmatrix}
0 \, 0 \, 1 \, 1 \, 0
\end{bmatrix}^T
$.
\end{definition}

\begin{figure}[H]
    \centering
    
    \subfigure[]{
        \label{subfig:a}
        \begin{tikzpicture}
        \node[transition,
            minimum size=0.5cm,
            label={above:$c$}
            ] (c) at (-6, 0) {};
        \node[place,
            minimum size=0.5cm,
            tokens=1,
            label={above:$1$}
            ] (1) at (-7.5,1) {};
        \node[place,
            minimum size=0.5cm,
            tokens=1,
            label={above:$2$}
            ] (2) at (-4.5,1) {};
        \node[transition,
            minimum size=0.5cm,
            label={right:$a$}
            ] (a) at (-7.5,0) {};
        \node[transition,
            minimum size=0.5cm,
            label={right:$b$}
            ] (b) at (-4.5,0) {};
        \node[transition,
            minimum size=0.5cm,
            label={above:$d$}
            ] (d) at (-6,-2.5) {};
        \node[place,
            minimum size=0.5cm,
            label={left:$5$}
            ] (5) at (-6,-3.5) {};
        \node[place,
            minimum size=0.5cm,
            label={right:$4$}
            ] (4) at (-4.5,-1) {};
        \node[place,
            minimum size=0.5cm,
            label={right:$3$}
            ] (3) at (-7.5,-1) {};
        \draw[thick] 
        (c) edge[post] (1)
        (c) edge[post] (2)
        (4) edge[post] (c)
        (4) edge[post] (d)
        (d) edge[post] (5)
        (3) edge[post] (c)
        (a) edge[post] (3)
        (1) edge[post] (a)
        (2) edge[post] (b)
        (b) edge[post] (4);
        \draw (d.180) edge[bend right=260,pre] (1);
    \end{tikzpicture}
    }
    \subfigure[]{
        \label{subfig:b}
        \begin{tikzpicture}
        \node[transition,
            minimum size=0.5cm,
            label={above:$c$}
            ] (c) at (-6, 0) {};
        \node[place,
            minimum size=0.5cm,
            tokens=0,
            label={above:$1$}
            ] (1) at (-7.5,1) {};
        \node[place,
            minimum size=0.5cm,
            tokens=1,
            label={above:$2$}
            ] (2) at (-4.5,1) {};
        \node[transition,
            minimum size=0.5cm,
            label={right:$a$}
            ] (a) at (-7.5,0) {};
        \node[transition,
            minimum size=0.5cm,
            label={right:$b$}
            ] (b) at (-4.5,0) {};
        \node[transition,
            minimum size=0.5cm,
            label={above:$d$}
            ] (d) at (-6,-2.5) {};
        \node[place,
            minimum size=0.5cm,
            label={left:$5$}
            ] (5) at (-6,-3.5) {};
        \node[place,
            minimum size=0.5cm,
            label={right:$4$}
            ] (4) at (-4.5,-1) {};
        \node[place,
            minimum size=0.5cm,
            tokens=1,
            label={right:$3$}
            ] (3) at (-7.5,-1) {};
        \draw[thick] 
        (c) edge[post] (1)
        (c) edge[post] (2)
        (4) edge[post] (c)
        (4) edge[post] (d)
        (d) edge[post] (5)
        (3) edge[post] (c)
        (a) edge[post] (3)
        (1) edge[post] (a)
        (2) edge[post] (b)
        (b) edge[post] (4);
        \draw (d.180) edge[bend right=260,pre] (1);
    \end{tikzpicture}
    }
    \subfigure[]{
        \label{subfig:c}
        \begin{tikzpicture}
        \node[transition,
            minimum size=0.5cm,
            label={above:$c$}
            ] (c) at (-6, 0) {};
        \node[place,
            minimum size=0.5cm,
            tokens=0,
            label={above:$1$}
            ] (1) at (-7.5,1) {};
        \node[place,
            minimum size=0.5cm,
            tokens=0,
            label={above:$2$}
            ] (2) at (-4.5,1) {};
        \node[transition,
            minimum size=0.5cm,
            label={right:$a$}
            ] (a) at (-7.5,0) {};
        \node[transition,
            minimum size=0.5cm,
            label={right:$b$}
            ] (b) at (-4.5,0) {};
        \node[transition,
            minimum size=0.5cm,
            label={above:$d$}
            ] (d) at (-6,-2.5) {};
        \node[place,
            minimum size=0.5cm,
            label={left:$5$}
            ] (5) at (-6,-3.5) {};
        \node[place,
            minimum size=0.5cm,
            tokens=1,
            label={right:$4$}
            ] (4) at (-4.5,-1) {};
        \node[place,
            minimum size=0.5cm,
            tokens=1,
            label={right:$3$}
            ] (3) at (-7.5,-1) {};
        \draw[thick] 
        (c) edge[post] (1)
        (c) edge[post] (2)
        (4) edge[post] (c)
        (4) edge[post] (d)
        (d) edge[post] (5)
        (3) edge[post] (c)
        (a) edge[post] (3)
        (1) edge[post] (a)
        (2) edge[post] (b)
        (b) edge[post] (4);
        \draw (d.180) edge[bend right=260,pre] (1);
    \end{tikzpicture}
    }
    \subfigure[]{
        \label{subfig:d}
        \begin{tikzpicture}
        \node[transition,
            minimum size=0.5cm,
            label={above:$c$}
            ] (c) at (-6, 0) {};
        \node[place,
            minimum size=0.5cm,
            tokens=1,
            label={above:$1$}
            ] (1) at (-7.5,1) {};
        \node[place,
            minimum size=0.5cm,
            tokens=1,
            label={above:$2$}
            ] (2) at (-4.5,1) {};
        \node[transition,
            minimum size=0.5cm,
            label={right:$a$}
            ] (a) at (-7.5,0) {};
        \node[transition,
            minimum size=0.5cm,
            label={right:$b$}
            ] (b) at (-4.5,0) {};
        \node[transition,
            minimum size=0.5cm,
            label={above:$d$}
            ] (d) at (-6,-2.5) {};
        \node[place,
            minimum size=0.5cm,
            label={left:$5$}
            ] (5) at (-6,-3.5) {};
        \node[place,
            minimum size=0.5cm,
            label={right:$4$}
            ] (4) at (-4.5,-1) {};
        \node[place,
            minimum size=0.5cm,
            label={right:$3$}
            ] (3) at (-7.5,-1) {};
        \draw[thick] 
        (c) edge[post] (1)
        (c) edge[post] (2)
        (4) edge[post] (c)
        (4) edge[post] (d)
        (d) edge[post] (5)
        (3) edge[post] (c)
        (a) edge[post] (3)
        (1) edge[post] (a)
        (2) edge[post] (b)
        (b) edge[post] (4);
        \draw (d.180) edge[bend right=260,pre] (1);
    \end{tikzpicture}
    }
    \caption{Esempio di un sistema e di una sua possibile evoluzione}
    \label{fig:sistema_el}
\end{figure}

E' possibile ricavare da un sistema il relativo \emph{grafo delle marcature 
raggiungibili}, un sistema di transizione etichettato in cui i vertici del
grafo rappresentano uno stato globale del sistema, e gli archi rappresentano
lo scatto di una o più transizioni.
\begin{definition}
\normalfont
Il grafo delle marcature raggiungibili è definito tramite una quadrupla:
$$ CG_{\Sigma} = (C_{\Sigma}, U_{\Sigma}, A, s_0)$$
In cui:\\
\begin{itemize}
    \item[] $CG_{\Sigma}$ è l'insieme delle \emph{marcature raggiungibili}
    dalla marcatura iniziale.
    \item[] $U_{\Sigma}$ = $\{ U \subseteq E \,|\, \exists \,m, m'
    \in C_{\Sigma} : m[U>m' \}$ detto l'insieme dei \emph{passi}.
    \item[] $A$ = $\{ (m, U, m') \,|\, m, m' \in C_{\Sigma} :
    U \in U_{\Sigma}, \,m[U>m' \}$ costituisce gli archi etichettati del 
    grafo.
    \item[] $s_0 = m_0$, la marcatura iniziale.
\end{itemize}

\end{definition}

\begin{figure}[H]
    \centering
    \begin{tikzpicture}
        \node[] (1_2) at (0,0) {\{1,2\}};
        \node[] (3_4) at (0,-3) {\{3,4\}}; 
        \node[] (3_2) at (3,-1.5) {\{3,2\}};
        \node[] (1_4) at (-3,-1.5) {\{1,4\}};
        \node[] (5) at (-6,-3) {\{5\}};
        \node[] (invisibile) at (-1, 1){};
        \draw[->] (invisibile) -- (1_2) ;
        \draw (1_2) -- (3_2) node [midway, above=2pt] {a};
        \draw[->] (3_2) -- (3_4) node [midway, above=2pt] {b} ;
        \draw[->] (1_4) -- (3_4) node [midway, above=2pt] {a} ;
        \draw[->] (1_2) -- (3_4) node [midway, right=1pt] {a,b};
        \draw[->] (1_4) -- (5) node [midway, above=2pt] {d};
        \draw[->] (1_2) -- (1_4) node [midway, above=2pt] {b};
    \end{tikzpicture}
    \caption{Grafo delle marcature raggiungibili relativo al sistema della figura \ref{fig:sistema_el}}
    \label{fig:grafo_casi}
\end{figure}

\subsection{Proprietà di comportamento}
Esistono alcune proprietà di interesse che riassumono gli aspetti salienti
del comportamento di un sistema, fra queste vi è la \emph{limitatezza},
l'\emph{assenza di deadlock}, la \emph{vivezza} e la \emph{reversibilità}.

\begin{definition}
\normalfont
Un sistema si dice \emph{limitato} $\iff$
$$\exists n \in \mathbb{N} : \forall p \in P, \forall m \in [m_0> : 
m(p) \leq n$$

cioè se non esiste un'evoluzione del sistema per cui un numero infinito
di marche possa accumularsi in un posto.
\end{definition}

\begin{definition}
\normalfont
Un sistema si dice \emph{deadlock-free} se per ciascuna marcatura
raggiungibile dalla marcatura iniziale, esiste almeno una transizione 
abilitata in tale marcatura, formalmente:
$$\forall m\ \in [m_0> \exists t \in T : m[t>$$

Un modo semplice, anche se non ideale, per dimostrare che un sistema sia
privo di deadlock consiste nel ricavare il grafo delle marcature
raggiungibili relativo al sistema e di verificare che non vi siano dei nodi
pozzo. Un nodo privo di archi uscenti rappresenta infatti una marcatura di
deadlock, come per esempio il nodo $\{ 5\}$ nel grafo della figura 
\ref{fig:grafo_casi}.
Questo metodo non è sempre applicabile: alcune reti possono avere un numero
di marcature raggiungibili infinito, in questi casi si è impossibilitati
a ricavare il relativo sistema di transizione.

Esistono sistemi che per loro natura devono avere un comportamento ciclico ed
eseguibile per un periodo di tempo indeterminato, si pensi per esempio ad un
web server. In questi tipi di sistema la presenza di una situazione 
di deadlock è indesiderata, e l'utilizzo delle tecniche di verifica
dei metodi formali possono essere uno strumento utile per fare 
luce su questi scenari prima della fase implementativa.

Oltre che l'analisi del grafo delle marcature raggiungibili, esistono
diverse tecniche che permettono di verificare la presenza
delle proprietà di comportamento. Fra queste vi sono le tecniche basate sull'algebra lineare:
una rete di Petri è codificabile in una matrice, detta matrice 
d'\emph{incidenza}, dalla cui analisi è possibile ricavare varie
informazioni utili sul comportamento dei sistemi basati su tale rete.
Esistono varie classi di reti di Petri su cui sono formulati dei teoremi 
che forniscono condizioni necessarie e sufficienti affinchè determinate
proprietà siano rispettate.
Lo stabilire l'appartenenza di una rete ad una determinata
classe rivela di per sé molte informazioni sul suo comportamento.

\end{definition}





